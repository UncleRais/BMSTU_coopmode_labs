\documentclass[12pt, a4paper]{article}

\usepackage[utf8]{inputenc}
\usepackage[T2A]{fontenc}
\usepackage[russian]{babel}
\usepackage[]{float}

\usepackage[oglav, boldsect, eqwhole, figwhole, %
   remarks, hyperref, hyperprint]{fn2kursstyle}

\frenchspacing
\righthyphenmin=2

%вставка рисунков
\newcommand{\Picture}[4]
{
\begin{figure}[H]
\noindent 
\centering\includegraphics[width = #2\textwidth]{pic/#1}
\caption{#3}
\label{#4}
\end{figure}
}

%Командна для римских прописных 
\newcommand{\RomanNumeralCaps}[1]
    {\MakeUppercase{\romannumeral #1}}

\title{Численное решение краевых задач для двумерного уравения Пуассона}
\group{ФН2-62Б}
\author{A.\,И.~Токарев}
\secauthor{Ю.\,А.~Сафронов}
\supervisor{}
\date{2021}

\def\hmath$#1${\texorpdfstring{{\rmfamily\textit{#1}}}{#1}}

\begin{document}
\maketitle
\tableofcontents 
\newpage

\section{Погрешность на точном решении}

Требуется рассмотреть погрешность разностной схемы на точном решении и проверить порядок аппроксимации. Рассмотрим первый тестовый пример:

\[
\]

\centerline{ЗДЕСЬ ПРИМЕР}

Проведем рассчет на равномерной сетке для разных шагов во времени и пространству, для этих решений построим таблицу с погрешностью на точном решении в норме $C$ пространства непрерывных функций:

\centerline{ЗДЕСЬ ТАБЛИЦА С ПОРЯДКАМИ}
%\Picture{погр.png}{1}{Значение рядом со стрелочкой показывает, во сколько раз уменьшилась ошибка или уменьшился шаг}{}


\newpage
\section{Контрольные вопросы}
\begin{enumerate}
\item Оцените число действий, необходимое для перехода на следующий слой по времени методом переменных направлений.

Чтобы расчитать прогонку для перехода к промежуточному временному слою нужно $5(N_2-1)$ действий, для перехода от промежуточного к следующему нужно еще $5(N_1-1)$ действий. Для подсчета правой части $F_{ij}^k, j = \overline{1,N_2-1}$, нужно $3(N_2 - 1)$ действий, для подсчета правой части $\hat{F}_{ij}^k, i = \overline{1,N_1-1}$, нужно еще $3(N_1 - 1)$ действий. Итого требуется
\[
(3 + 5)(N_2-1) + (3 + 5)(N_1-1) = 8(N_2 + N_1) - 16
\]
действий (без учета подсчета $\dfrac{1}{h_1^2}$, $\dfrac{1}{h_2^2}$ и т.д.).

\item Почему при увеличении числа измерений резко возрастает количество операций для решения неявных схем (по сравнению с одномерной схемой)?

Потому что приходится многограткно увеличивать количество решений СЛАУ. В одномерном случае СЛАУ нужно считать вдоль одной переменной, в двумерном дволь двух и т.д.

\item Можно ли использовать метод переменных направлений в
областях произвольной формы?

Да. Форма области влияет лишь на размерность трехдиагональной матрицы на каждой итерации. 


\item Можно ли использовать метод переменных направлений для решения пространственных и вообще $n$-мерных задач?

Нет. Перенести ее на 3-х мерный случай не удалось до сих пор. Существует эволюционно-факторизованная схема, которую можном трактовать как обобщение продольно-поперечной схемы(метод переменных направлений) на трехмерный случай (и даже на случай произвольного числа измерений).

\item Можно ли использовать метод переменных направлений на неравномерных сетках?

Сложно себе представить составление трехдиагональной матрицы вдоль какого-либо направления на неравномерной сетке, поэтому нет.
\end{enumerate}


\newpage

\end{document} 
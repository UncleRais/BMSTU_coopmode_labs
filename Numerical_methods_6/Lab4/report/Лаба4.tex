\documentclass[12pt, a4paper]{article}

\usepackage[utf8]{inputenc}
\usepackage[T2A]{fontenc}
\usepackage[russian]{babel}
\usepackage[]{float}

\usepackage[oglav, boldsect, eqwhole, figwhole, %
   remarks, hyperref, hyperprint]{fn2kursstyle}

\frenchspacing
\righthyphenmin=2

%вставка рисунков
\newcommand{\Picture}[4]
{
\begin{figure}[H]
\noindent 
\centering\includegraphics[width = #2\textwidth]{pic/#1}
\caption{#3}
\label{#4}
\end{figure}
}

%Командна для римских прописных 
\newcommand{\RomanNumeralCaps}[1]
    {\MakeUppercase{\romannumeral #1}}

\title{Численное решение краевых задач для двумерного уравения Пуассона}
\group{ФН2-62Б}
\author{A.\,И.~Токарев}
\secauthor{Ю.\,А.~Сафронов}
\supervisor{}
\date{2021}

\def\hmath$#1${\texorpdfstring{{\rmfamily\textit{#1}}}{#1}}

\begin{document}
\maketitle
\tableofcontents 
\newpage

\section{Погрешность на точном решении}

Требуется рассмотреть погрешность разностной схемы на точном решении и проверить порядок аппроксимации. Рассмотрим первый тестовый пример:

\[
\]

\centerline{ЗДЕСЬ ПРИМЕР}

Проведем рассчет на равномерной сетке для разных шагов во времени и пространству, для этих решений построим таблицу с погрешностью на точном решении в норме $C$ пространства непрерывных функций:

\centerline{ЗДЕСЬ ТАБЛИЦА С ПОРЯДКАМИ}
%\Picture{погр.png}{1}{Значение рядом со стрелочкой показывает, во сколько раз уменьшилась ошибка или уменьшился шаг}{}


\newpage
\section{Контрольные вопросы}
\begin{enumerate}
\item Оцените число действий, необходимое для перехода на следующий слой по времени методом переменных направлений.

\item Почему при увеличении числа измерений резко возрастает количество операций для решения неявных схем (по сравнению с одномерной схемой)?

\item Можно ли использовать метод переменных направлений в
областях произвольной формы?

\item Можно ли использовать метод переменных направлений для решения пространственных и вообще $n$-мерных задач?

\item Можно ли использовать метод переменных направлений на неравномерных сетках?

\end{enumerate}


\newpage

\end{document} 
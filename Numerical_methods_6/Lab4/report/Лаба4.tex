\documentclass[12pt, a4paper]{article}

\usepackage[utf8]{inputenc}
\usepackage[T2A]{fontenc}
\usepackage[russian]{babel}
\usepackage[]{float}

\usepackage[oglav, boldsect, eqwhole, figwhole, %
   remarks, hyperref, hyperprint]{fn2kursstyle}

\frenchspacing
\righthyphenmin=2

%вставка рисунков
\newcommand{\Picture}[4]
{
\begin{figure}[H]
\noindent 
\centering\includegraphics[width = #2\textwidth]{pic/#1}
\caption{#3}
\label{#4}
\end{figure}
}

%Командна для римских прописных 
\newcommand{\RomanNumeralCaps}[1]
    {\MakeUppercase{\romannumeral #1}}

\title{Численное решение краевых задач для двумерного уравения Пуассона}
\group{ФН2-62Б}
\author{A.\,И.~Токарев}
\secauthor{Ю.\,А.~Сафронов}
\supervisor{}
\date{2021}

\def\hmath$#1${\texorpdfstring{{\rmfamily\textit{#1}}}{#1}}

\begin{document}
\maketitle
\tableofcontents 
\newpage

\section{Погрешность на точном решении}
\subsection{Пример №1}

Требуется рассмотреть погрешность разностной схемы на точном решении и проверить порядок аппроксимации. Рассмотрим второй тестовый пример:

\[
\Delta u = 0, \quad (x_1, x_2)\in G = [0,1] \times[0,1],
\]
\[
\dfrac{\partial u}{\partial n}(x_1, 0) = -1, \quad \dfrac{\partial u}{\partial n}(x_1, 1) = 1,
\]
\[
u(0, x_2) = 1+x_2, \quad u(0,x_2) = 1+x_2.
\]

Точное решение :
\[
u(x_1,x_2) = 1 + x_2.
\]
%\centerline{ЗДЕСЬ ПРИМЕР}

Проведем рассчет на равномерной сетке для разных шагов по пространственным координатам, для этих решений построим таблицу с погрешностью на точном решении в норме $C$ пространства непрерывных функций:

%\centerline{ЗДЕСЬ ТАБЛИЦА С ПОРЯДКАМИ}
\Picture{погр.png}{1}{Значение рядом со стрелочкой показывает, во сколько раз уменьшилась ошибка или уменьшился шаг}{}

\Picture{точн.pdf}{1}{График точного решения}{}

\Picture{числ.pdf}{1}{График численного решения при $N_1 = 30$, $N_2=30$ ($h_1 =h_2= \frac{1}{30}$)}{}

%\subsection{Пример №2}
% Рассмотрим третий тестовый пример:

%\[
%\Delta u = 4, \quad (x_1, x_2)\in G = [0,1] \times[0,1],
%\]
%\[
%\dfrac{\partial u}{\partial n}(0, x_2) = 0, \quad \dfrac{\partial u}{\partial n}(1, x_2) = 1,
%\]
%\[
%u(x_1, 0) = {x_1}^2, \quad u(x_1,1) ={x_1}^2.
%\]

%Точное решение :
%\[
%u(x_1,x_2) = {x_1}^2 + {x_2}^2.
%\]


\newpage
\section{Контрольные вопросы}
\begin{enumerate}
\item Оцените число действий, необходимое для перехода на следующий слой по времени методом переменных направлений.

Чтобы расчитать прогонку для перехода к промежуточному временному слою нужно $5(N_2-1)$ действий, для перехода от промежуточного к следующему нужно еще $5(N_1-1)$ действий. Для подсчета правой части $F_{ij}^k, j = \overline{1,N_2-1}$, нужно $3(N_2 - 1)$ действий, для подсчета правой части $\hat{F}_{ij}^k, i = \overline{1,N_1-1}$, нужно еще $3(N_1 - 1)$ действий. Итого требуется
\[
(3 + 5)(N_2-1)(N_1-1) + (3 + 5)(N_1-1)(N_2 - 1) = 16(N_2-1)(N_1-1) = O(N_1N_2)
\]
действий (без учета подсчета $\dfrac{1}{h_1^2}$, $\dfrac{1}{h_2^2}$ и т.д.).

\item Почему при увеличении числа измерений резко возрастает количество операций для решения неявных схем (по сравнению с одномерной схемой)?

Потому что матрица системы сильно увеличивается.

\item Можно ли использовать метод переменных направлений в
областях произвольной формы?

Да. Форма области влияет лишь на размерность трехдиагональной матрицы на каждой итерации. 


\item Можно ли использовать метод переменных направлений для решения пространственных и вообще $n$-мерных задач?

Нет. Существует локально-одномерная схема, которую можном трактовать как обобщение продольно-поперечной схемы(метод переменных направлений) на трехмерный случай (и даже на случай произвольного числа измерений). 

\item Можно ли использовать метод переменных направлений на неравномерных сетках?

Да, можно, но только на прямоугольных.
\end{enumerate}


\newpage

\end{document} 
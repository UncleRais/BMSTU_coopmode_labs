\documentclass[12pt, a4paper]{article}

\usepackage[utf8]{inputenc}
\usepackage[T2A]{fontenc}
\usepackage[russian]{babel}
\usepackage[]{float}

\usepackage[oglav, boldsect, eqwhole, figwhole, %
   remarks, hyperref, hyperprint]{fn2kursstyle}

\frenchspacing
\righthyphenmin=2

%вставка рисунков
\newcommand{\Picture}[4]
{
\begin{figure}[H]
\noindent 
\centering\includegraphics[width = #2\textwidth]{pic/#1}
\caption{#3}
\label{#4}
\end{figure}
}

%Командна для римских прописных 
\newcommand{\RomanNumeralCaps}[1]
    {\MakeUppercase{\romannumeral #1}}

\newcommand{\norm}[1]{\left\lVert#1\right\rVert}

\title{Численные решения интегральных уравнений}
\group{ФН2-62Б}
\author{A.\,И.~Токарев}
\secauthor{Ю.\,А.~Сафронов}
\supervisor{}
\date{2022}

\def\hmath$#1${\texorpdfstring{{\rmfamily\textit{#1}}}{#1}}

\begin{document}
    \maketitle
    \tableofcontents 

    \section{Контрольные вопросы}

    \begin{enumerate}
    \item При выполнении каких условий интегральное уравнение Фредгольма 2 рода имеет решение? В каком случае решение является единственным?
    На вопросы существования решения этого уравнения отвечает классическая теория Фредгольма.
    Если $ K(t, s),\,f(t) $ -- непрерывные функции на заданных отрезках, то $\forall \lambda \in \mathbb{R}$ интегральное уравнение. Рассмотрим интегральное уравнение 
    \[
        u(t) - \lambda \displaystyle \int_a^b K(t, s) u(s) ds = f(t),
    \]
    \noindent где $ K(t, s) $ -- ядро этого уравнения, являющееся функцией непрерывной на декартовом произведении отрезка $[a, b]$ на себя, а $ \lambda \neq 0$ -- параметр данного уравнения. Представим уравнение в операторном виде:
    \[
      (I - A)(u) = f,  
    \]
    \noindent где оператор $A$ преобразует исходную функцию $ u(t) $ в 
    \[
      v(t) = \lambda \displaystyle \int_a^b K(t, s)u(s)ds,  
    \]
    \noindent то есть действует из $C[a,b] \rightarrow C[a, b]$. Используя свойства определенного интеграла для функции $ v = A(u) $ и любой точки $ t \in [a, b] $ находим:
    \[
      | v(t) | = | \lambda \displaystyle \int_a^b K(t, s)u(s) ds | \leq | \lambda | \displaystyle \int_a^b | K(t, s) u(s) | ds \leq |\lambda|(b-a) \max_{t, s \in [a, b]} | K(t, s) | \max_{t \in [a, b]} |u(t)|.
    \]
    Пусть $ q = | \lambda |(b-a) \displaystyle \max_{t, s \in [a, b]} | K(t, s) | < 1  \Rightarrow $
    \[
      \norm{ Au }_{C[a, b]} = \max_{t \in [a, b]} | v(t) | \leq q \max_{ t \in [a, b] } | u(t) | = q \norm{u}_{C[a, b]}.
    \]
    Таким образом $ \norm{A} \leq q < 1 $. Значит, согласено теореме об обратном операторе $ \exists S = (I - A)^{-1} $, то есть рассматриваемое интегральное уравнение имеет единственное решение $u^0 = S(f)$, причем 
    \[
      | u^0 |_{C[a, b]} \leq \norm{S} \norm{f}_{C[a, b]} \leq \dfrac{1}{1 - q} \norm{f}_{C[a, b]}, \quad \norm{f}_{C[a, b]} = \max_{t \in [a, b]} |f(t)|.
    \]
    Дополнение: если однородное уравнение $ f(x) = 0 $ имеет только тривиальное решение, то значение параметра $ \lambda $ называется правильным или регулярным. Тогда у неоднородного уравнения при любой правой части $ f(x) $ существует единственное решение.

    \item Можно ли привести матрицу СЛАУ, получающуюся при использовании метода квадратур, к симметричному виду в случае, если ядро интегрального уравнения является симметричным, т. е. $ K(x, s) = K(s, x) $?

    \item Предложите способ контроля точности результата вычислений при использовании метода квадратур.
    Можно сравнивать с точным решением.

    \item Оцените возможность и эффективность применения методов квадратур, простой итерации и замены ядра при решении интегральных уравнений Вольтерры 2 рода.

    \item Что называют резольвентой ядра интегрального уравнения?
    Резольвентой интегрального уравнения Фредгольма 2 рода называется такая функция $ R = R(s, t, \lambda) $, что решение этого уравнения представляется в виде:
    \[
      u(s) = f(s) + \displaystyle \int_a^b R(s, t, \lambda) f(t) dt,
    \]
    где $\lambda$ не является собственным числом.
    \item Почему замену ядра интегрального уравнения вырожденным предпочтительнее осуществлять путем разложения по многочленам Чебышева, а не по формуле Тейлора?

    \item Какие вы можете предложить методы решения переопределенной системы $ (5.13) $, $ (5.17) $ помимо введения дополнительно переменной $ R $?
    \end{enumerate}

    \newpage

\end{document} 
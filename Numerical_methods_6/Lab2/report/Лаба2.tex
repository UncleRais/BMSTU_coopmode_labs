\documentclass[12pt, a4paper]{article}

\usepackage[utf8]{inputenc}
\usepackage[T2A]{fontenc}
\usepackage[russian]{babel}
\usepackage[]{float}

\usepackage[oglav, boldsect, eqwhole, figwhole, %
   remarks, hyperref, hyperprint]{fn2kursstyle}

\frenchspacing
\righthyphenmin=2

%Командна для римских прописных 
\newcommand{\RomanNumeralCaps}[1]
    {\MakeUppercase{\romannumeral #1}}

\title{Численное решение краевых задач для одномерного уравнения теплопроводности}
\group{ФН2-62Б}
\author{A.\,И.~Токарев}
\secauthor{Ю.\,А.~Сафронов}
\supervisor{}
\date{2021}

\def\hmath$#1${\texorpdfstring{{\rmfamily\textit{#1}}}{#1}}

\begin{document}
\maketitle
\tableofcontents 
\newpage

\section{Исходные данные}



\newpage
\section{Контрольные вопросы}
\begin{enumerate}
\item Дайте определения терминам: корректно поставленная задача, понятие аппроксимации дифференциальной задачи разностной схемой, порядок аппроксимации, однородная схема, консервативная схема, монотонная схема, устойчивая разностная схема (условно/абсолютно), сходимость.

Задача называется корректной по Адамару в том случае, если её решение существует, единственно и непрерывно зависит от входных данных. Разностная схема аппроксимирует исходную задачу, если порядок аппроксимации РС стремится к нулю по норме при дроблении шага. РС называется однородной, если её уравнение записано одинаково и на одном шаблоне во всех узлах сетки без явного выделения особенности. РС называется консервативной, если для её решения выполняются законы сохранения. РС называется монотонной, если в одном случае её решение сохраняет монотонность по пространственной переменной при условии, что такое свойство справедливо для исходной задачи. РС называется устойчивой, если её решение непрерывно зависит от входных данных. Устойчивость называется безусловной, если её наличие или отсуствие не зависит от соотношения между шагами, иначе условной. СХОДИМОСТЬ = УСТОЙЧИВОСТЬ + ПОРЯДОК АППРОКСИМАЦИИ.

\item Какие из рассмотренных схем являются абсолютно устойчивыми? Какая из рассмотренных схем позволяет вести расчеты с более крупным шагом по времени?

Смешанная разностная схема, определяемая параметром $\sigma$, устойчива, если
\[
\sigma \geq \dfrac{1}{2} - \dfrac{c \rho h^2}{4 \tau \max K(x)}.
\]

Отсюда следует, что смешанная схема с $\sigma > \sfrac{1}{2}$, в частности,
неявная схема устойчивы при любых соотношениях параметров $\tau$ и $h$, то есть устойчивы абсолютно. Симметричная схема (при $\sigma = \sfrac{1}{2}$), так как она позволяет достичь второго порядка аппроксимации. 

\item Будет ли смешанная схема иметь второй порядок аппроксимации при $a_i = \dfrac{2 K(x_i)K(x_{i-1})}{K(x_i) + K(x_{i-1})}$?

\[
a_i = \left(\int\limits_{x_{i-1}}^{x_{i}}{\dfrac{1}{h}\dfrac{dx}{K(x)}}\right)^{-1} \approx \left(\dfrac{1}{h}\dfrac{h}{2}\left(\dfrac{1}{K(x_{i})} + \dfrac{1}{K(x_{i-1})}\right)\right)^{-1} = 
\]
\[
\dfrac{2 K(x_i)K(x_{i-1})}{K(x_i) + K(x_{i-1})},
\]
следовательно, поскольку формула трапеций имеет второй порядок аппроксимации, то и схема будет иметь второй порядок аппроксимации. 

\item Какие методы (способы) построения разностной аппроксимации граничных условий с порядком точности $O(\tau + h^2)$, $O(\tau^2 +h^2)$ и $O(\tau^2 + h)$ вы знаете?

При $\sigma = 1$ (полностью неявная схема) порядок первый по времени и второй по пространству, при $\sigma  = 1/2$ (симметричная схема) порядок второй и по пространству и по времени. Чтобы добиться первого порядка по времени и второго по пространству, нужно взять симметричную схему, так как другие не дают второй порядок по времени, и <<испортить>> порядок по пространству, взяв квадратурные формулы первого порядка для подсчета коэффициентов $a_i$.

\item При каких $h$, $\tau$ и $\sigma$ смешанная схема монотонна? Проиллюстрируйте результатами расчетов свойства монотонных и немонотонных разностных схем.

Любую разностную схему можно записать в виде:
\[
  A(x)y(x) = \displaystyle\sum_{\xi \in S^'(x)} B(x, \xi)y(\xi) + F(x), \quad x\in G_h,
\]
\noindent где $S$ -- шаблон разностной схемы. Определим сеточный оператор:
\[
  Ly = D(x)y(x) - \displaystyle\sum_{\xi \in S^'(x)} B(x, \xi)(y(\xi) - y(x)), \quad D = A(x) - \displaystyle\sum_{\xi \in S^'(x)} B(x, \xi) \Rightarrow Ly = F.
\]
Тогда для монотонности необходимо, чтобы сетки $G_h, w$(множество внутренних узлов сетки) были связны, а коэффициенты $A, B, D$ положительны. В этом случае функция $y$ не может принимать своего максимального или минимального значения, то есть она монотонна(следует из принципа максимума).

\[
\hat{y_n} = \sum_{i}^{n}{\alpha_i y_{n+i}}
\]
Явная двухслойная линейная однородная схема с постоянными коэффициентами монотонна в смысле сохранения монотонности профиля $y$ тогда и только тогда, когда все коэффициенты $\alpha_i \geq 0$.

\item Какие ограничения на $h$, $\tau$ и $\sigma$ накладывают условия устойчивости прогонки?
Прогоночные коэффициенты выглядят следующим образом:
\[
    A_i = \dfrac{\sigma}{h}a_i, \quad B_i = \dfrac{\sigma}{h}a_{i+1}, \quad C_i = \dfrac{\sigma}{h}a_i + \dfrac{\sigma}{h}a_{i+1} + c \rho \dfrac{h}{\tau},
\]
\noindent откуда следует, что $ C_i > A_i + B_i $, когда $ \sigma \geq 0, h > 0, t > 0 $, а значит прогонка устойчива.

\item В случае $K=K(u)$ чему равно количество внутренних итераций, если итерационный процесс вести до сходимости, а не обрывать после нескольких первых итераций?

\item Для случая $K = K(u)$ предложите способы организации внутреннего итерационного процесса или алгоритмы, заменяющие его.
Для органазции внутреннего итерационного процесса подходит любой гибридный метод. Например, линейное приближение $a_i$ можно получать с помощью метода Ньютона на внешних итерациях, а уточнять значение на внутренних с помощью метода простой итерации, Якоби, Зейделя, минимальных невязок и наискорейшего спуска. Также может подойти обычный метод Ньютона решения нелинейного уравнения.  

\end{enumerate}


\newpage

\end{document} 
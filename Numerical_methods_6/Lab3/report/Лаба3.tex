\documentclass[12pt, a4paper]{article}

\usepackage[utf8]{inputenc}
\usepackage[T2A]{fontenc}
\usepackage[russian]{babel}
\usepackage[]{float}

\usepackage[oglav, boldsect, eqwhole, figwhole, %
   remarks, hyperref, hyperprint]{fn2kursstyle}

\frenchspacing
\righthyphenmin=2

%вставка рисунков
\newcommand{\Picture}[4]
{
\begin{figure}[H]
\noindent 
\centering\includegraphics[width = #2\textwidth]{pic/#1}
\caption{#3}
\label{#4}
\end{figure}
}

%Командна для римских прописных 
\newcommand{\RomanNumeralCaps}[1]
    {\MakeUppercase{\romannumeral #1}}

\title{Численное решение краевых задач для одномерного волнового уравнения}
\group{ФН2-62Б}
\author{A.\,И.~Токарев}
\secauthor{Ю.\,А.~Сафронов}
\supervisor{}
\date{2021}

\def\hmath$#1${\texorpdfstring{{\rmfamily\textit{#1}}}{#1}}

\begin{document}
\maketitle
\tableofcontents 
\newpage

\section{Исходные данные}

\newpage
\section{Контрольные вопросы}
\begin{enumerate}
\item Предложите разностные схемы, отличные от схемы <<крест>>,
для численного решения задачи.

Схема "крест":

\[
\dfrac{1}{\tau^2}(y_{n}^{j+1}- 2y_{n}^{j}+y_{n}^{j-1}) = \dfrac{a^2}{h^2}(y_{n+1}^{j}-2y_{n}^{j}+y_{n-1}^{j}).
\]

Схема аналогичная схеме Дюфорта - Франкела для уравнения теплопроводности: 

\[
\dfrac{1}{\tau^2}(y_{n}^{j+1}-(y_{n+1}^{j} + y_{n-1}^{j}) +y_{n}^{j-1}) = \dfrac{a^2}{h^2}(y_{n+1}^{j}-(y_{n}^{j+1} + y_{n}^{j-1})+y_{n-1}^{j}).
\]

В этой схеме мы просто заменили значение точки в средине "креста" на полусумму двух соседних точек. 


\item Постройте разностную схему с весами для уравнения колебаний струны. Является ли такая схема устойчивой и монотонной?

Конструировать схему можно из следующих соображений. Вторую производную по времени мы будем аппроксимировать как обычно, то есть производная назад от производных вперед или наоборот. Вторую производную по пространству можно искать как комбинацию вторых разностных производных с весами на разных временных слоях:

\[
y_{\bar{t}t} - a^2(\sigma \hat{y}_{\bar{x}x} + (1 - 2\sigma) y_{\bar{x}x} + \sigma  \check{y}_{\bar{x}x}) = 0.
\]

Данная схема имеет второй порядок апроксимации и по пространству и по времени. Эта схема является устойчивой при выполнении условия 
\[
\sigma \geqslant \dfrac{1}{4} - \dfrac{h^2}{4c^2\tau^2}
\]
и не является монотонной.

\item Предложите способ контроля точности полученного решения.

\item Приведите пример трехслойной схемы для уравнения теплопроводности. Как реализовать вычисления по такой разностной схеме? Является ли эта схема устойчивой?

Схема "крест" подходит для решения уравнения теплопроводности.


\end{enumerate}


\newpage

\end{document} 
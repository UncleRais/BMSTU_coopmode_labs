\documentclass[12pt, a4paper]{article}

\usepackage[utf8]{inputenc}
\usepackage[T2A]{fontenc}
\usepackage[russian]{babel}
\usepackage[]{float}

\usepackage[oglav, boldsect, eqwhole, figwhole, %
   remarks, hyperref, hyperprint]{fn2kursstyle}

\frenchspacing
\righthyphenmin=2

%вставка рисунков
\newcommand{\Picture}[4]
{
\begin{figure}[H]
\noindent 
\centering\includegraphics[width = #2\textwidth]{pic/#1}
\caption{#3}
\label{#4}
\end{figure}
}

%Командна для римских прописных 
\newcommand{\RomanNumeralCaps}[1]
    {\MakeUppercase{\romannumeral #1}}

\title{Численное решение краевых задач для одномерного волнового уравнения}
\group{ФН2-62Б}
\author{A.\,И.~Токарев}
\secauthor{Ю.\,А.~Сафронов}
\supervisor{}
\date{2021}

\def\hmath$#1${\texorpdfstring{{\rmfamily\textit{#1}}}{#1}}

\begin{document}
\maketitle
\tableofcontents 
\newpage

\section{Погрешность на точном решении}

Требуется рассмотреть погрешность разностной схемы на точном решении и проверить порядок аппроксимации. Рассмотрим первый тестовый пример:
\[
u^{''}_{tt}=u^{''}_{xx},\quad u^{'}_t(x,0) = 0 = \psi(x), \quad u(x,0) = \sin (\pi x) = f(x),\quad 0<x<1,
\]
\[
u(0,t) = 0, \quad, u(1,t) = 0,
\]
где точное решение ищется по формуле Д'Аламбера

\[
u(x,t) =\dfrac{ f(x - at) + f(x + at)}{2} + \dfrac{1}{2a}\int\limits_{x-at}^{x+at}{\psi (x)dx} = \dfrac{\sin(\pi(x-t)) + \sin(\pi(x + t))}{2} =
\]
\[
= \sin(\pi x)\cos(\pi t).
\]

Проведем рассчет на равномерной сетке для разных шагов во времени и пространству, для этих решений построим таблицу с погрешностью на точном решении в норме $C$ пространства непрерывных функций:
\Picture{погр.png}{1}{Значение рядом со стрелочкой показывает, во сколько раз уменьшилась ошибка или уменьшился шаг}{}

Видно, что с уменьшением шага по времени и пространству в 2 раза погрешность уменьшилась примерно в 4 раза, следовательно, разностная схема имеет 2-ой порядок по времени и по пространству.


\newpage
\section{Контрольные вопросы}
\begin{enumerate}
\item Предложите разностные схемы, отличные от схемы <<крест>>,
для численного решения задачи.

Схема "крест":

\[
\dfrac{1}{\tau^2}(y_{n}^{j+1}- 2y_{n}^{j}+y_{n}^{j-1}) = \dfrac{a^2}{h^2}(y_{n+1}^{j}-2y_{n}^{j}+y_{n-1}^{j}).
\]

Схема аналогичная схеме Дюфорта - Франкела для уравнения теплопроводности: 

\[
\dfrac{1}{\tau^2}(y_{n}^{j+1}-(y_{n+1}^{j} + y_{n-1}^{j}) +y_{n}^{j-1}) = \dfrac{a^2}{h^2}(y_{n+1}^{j}-(y_{n}^{j+1} + y_{n}^{j-1})+y_{n-1}^{j}).
\]

В этой схеме мы просто заменили значение точки в средине "креста" на полусумму двух соседних точек. 


\item Постройте разностную схему с весами для уравнения колебаний струны. Является ли такая схема устойчивой и монотонной?

Конструировать схему можно из следующих соображений. Вторую производную по времени мы будем аппроксимировать как обычно, то есть производная назад от производных вперед или наоборот. Вторую производную по пространству можно искать как комбинацию вторых разностных производных с весами на разных временных слоях:

\[
y_{\bar{t}t} - a^2(\sigma \hat{y}_{\bar{x}x} + (1 - 2\sigma) y_{\bar{x}x} + \sigma  \check{y}_{\bar{x}x}) = 0.
\]

Данная схема имеет второй порядок апроксимации и по пространству и по времени. Эта схема является устойчивой при выполнении условия 
\[
\sigma \geqslant \dfrac{1}{4} - \dfrac{h^2}{4c^2\tau^2}
\]
и не является монотонной.

\item Предложите способ контроля точности полученного решения.

Контролировать точность решения можно с помощью дробления шага по времени и пространтсву в течение всего алгоритма. Для этого можно вести пересчет в какой-то пространственной подобласти или сразу на всем отрезке в момент времени $\tau/2$, $\tau/4$ и т. д. с шагом $h/2$. Это может быть очень трудоемко, поэтому можно производить корректировку шага раз в 10 итераций. Причем уменьшать шаг по пространству проще, потому что не нужно еще раз считать значения в "среднем"  временном слое в точках, в которых значения уже посчитаны, достаточно лишь досчитать значения в промежуточных точках, которые появятся при дроблении. Основная проблема заключается в том, что схема трехслойная и использовать предыдущие значения не получится, так как они посчитаны с бОльшим шагом, а чтобы их пересчитать нужно запустить весь алгоритм заново. Возможно, имеет смысл заменить половинный предыдущий временной слой полусуммой "среднего" и "нижнего".

\item Приведите пример трехслойной схемы для уравнения теплопроводности. Как реализовать вычисления по такой разностной схеме? Является ли эта схема устойчивой?

Схема "крест" подходит для решения уравнения теплопроводности.


\end{enumerate}


\newpage

\end{document} 
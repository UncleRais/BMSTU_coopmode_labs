\documentclass[12pt, a4paper]{article}

\usepackage[utf8]{inputenc}
\usepackage[T2A]{fontenc}
\usepackage[russian]{babel}
\usepackage[]{float}

\usepackage[oglav, boldsect, eqwhole, figwhole, %
   remarks, hyperref, hyperprint]{fn2kursstyle}

\frenchspacing
\righthyphenmin=2

%Командна для римских прописных 
\newcommand{\RomanNumeralCaps}[1]
    {\MakeUppercase{\romannumeral #1}}

\title{Методы численного решения обыкновенных дифференциальных уравнений}
\group{ФН2-62Б}
\author{A.\,И.~Токарев}
\secauthor{Ю.\,А.~Сафронов}
\supervisor{}
\date{2021}

\def\hmath$#1${\texorpdfstring{{\rmfamily\textit{#1}}}{#1}}

\begin{document}
\maketitle
\tableofcontents 
\newpage

\section{Исходные данные}
Модель Лотки - Вольтерры динамики системы <<хищник-жертва>>
\begin{equation*}
\begin{cases}
\sfrac{dx_1}{dt}= x_1^{'} =r_1x_1 - b_{11}x_1^2-b_{12}x_1x_2\\
\sfrac{dx_2}{dt}= x_2^{'} =-r_2x_2-b_{22}x_2^2+b_{21}x_1x_2\\
r_1 = 0{.}4, \quad r_2 = 0{.}1, \quad b_{11} = 0{.}05,\\
b_{12} = 0{.}1,\quad b_{21} = 0{.}08,\quad b_{22} = 0{.}003,\\
t = 0...150,\\
x_1(0) = 1{.}0,\quad x_2(0) = 4{.}0.
\end{cases}
\end{equation*}

\section{Результаты расчетов}
\newpage

\section{Анализ результатов}
\newpage

\section{Контрольные вопросы}
\begin{enumerate}
\item Сформулируйте условия существования и единственности решения задачи Коши для обыкновенных дифференциальных уравнений. Выполнены ли они для вашего варианта задания?
\begin{equation*}
\begin{cases}
y^{'} = f(x,y)\\
y(x_0) = y_0\\
\end{cases}
\end{equation*}
Для того, чтобы существовало решение задачи Коши достаточно, чтобы функция $f(x,y)$ была непрерывна в ограниченной замкнутой области $G = \{(x,y) \in \mathbb{R}\colon |x - x_0|\leq a, |y - y_0|\leq b\}$. Чтобы решение было единственным, должно выполняться условие Липшица по правому аргументу (или должна существовать непрерывная частная производная по $y$), т.е.
\[
\forall (x,y_1), (x,y_2)\, \exists L>0 \colon |f(x,y_1) - f(x,y_2)|\leq L|y_1 - y_2|.
\] 
Для системы
\begin{equation*}
\begin{cases}
x_1^{'} = f_1(t,x_1,x_2)\\
x_2^{'} = f_2(t,x_1,x_2)\\
x_1(t_0) =x_{10},\quad x_2(t_0) = x_{20}, \\
\end{cases}
\end{equation*}
если в области $G \subset \mathbb{R}^3$ функции $f_1,f_2$  непрерывны и имеют непрерывные частные производные по $x_1, x_2$ , то в некотором интервале  существует единственное решение системы, удовлетворяющее начальным условиям.

Для модели Лотки - Вольтерры:
\begin{equation*}
\begin{cases}
f_1 =r_1x_1 - b_{11}x_1^2-b_{12}x_1x_2;\\
f_2 =-r_2x_2-b_{22}x_2^2+b_{21}x_1x_2;\\
\sfrac{\partial f_1}{\partial x_1} = r_1 - 2b_{11}x_1 - b_{12}x_2;\quad
\sfrac{\partial f_1}{\partial x_2} = -b_{12}x_1;\\
\sfrac{\partial f_2}{\partial x_1} =b_{12}x_2;\quad
\sfrac{\partial f_2}{\partial x_2} =-r_2 - 2b_{22}x_2 + b_{21}x_1;\\
\end{cases}
\end{equation*}
функции непрерывны всюду на $\mathbb{R}^3$, точка $(0, 1{.}0, 4{.}0) \in \mathbb{R}^3$, значит существует единственное решение задачи Коши.


\item Что такое фазовое пространство? Что называют фазовой траекторией? Что называют интегральной кривой? 

Фазовое пространство --- это пространство, каждая точка которого соответствует одному состоянию из множества всех возможных состояний системы.

Траекторию движения в фазовом пространстве называют фазовой траекторией.

Интегральной кривой называется график решения дифференциального уравнения. Фазовая траектория является проекцией интегральной кривой.

\item Каким порядком аппроксимации и точности обладают методы, рассмотренные в лабораторной работе?
\begin{enumerate}
\item Явный метод Эйлера --- первый порядок 
\item Неявный метод Эйлера --- второй порядок
\item Симметричная схема --- первый порядок
\item Метод РК 2-го порядка --- второй порядок
\item Метод РК 4-го порядка --- четвертый порядок
\item Метод прогноза и коррекции --- четвертый порядок
\end{enumerate}
\item Какие задачи называются жесткими? Какие методы предпочтительны для их решения? Какие из рассмотренных методов можно использовать для решения жестких задач?

Система ОДУ $y^' = Ay$ с постоянной матрицей $A$ называется жесткой, если все собственные числа $A=A_{n\times n}$ имеют отрицательную действительную часть ($\text{Re}\, \lambda_i < 0, i=1,n$), причем число $S = \sfrac{\max|\text{Re}\, \lambda|}{\min|\text{Re}\, \lambda|}$, называемое числом жесткости, велико.
Для решения жестких задач используются неявные методы, так как они обладают лучшими свойствами устойчивости. Из рассмотренных можно использовать неявный метод Эйлера, симметричную схему.

\item Как найти $\vec{y}_1, \vec{y}_2, \vec{y}_3$, чтобы реализовать алгоритм прогноза и коррекции?
\item Какой из рассмотренных алгоритмов является менее трудоемким? Какой из рассмотренных алгоритмов позволяет достигнуть заданную точность, используя наибольший шаг интегрирования? Какие достоинства и недостатки рассмотренных алгоритмов вы можете указать?
\item Какие алгоритмы, помимо правила Рунге, можно использовать для автоматического выбора шага?

\end{enumerate}
\newpage

\end{document} 
\documentclass[12pt, a4paper]{article}

\usepackage[utf8]{inputenc}
\usepackage[T2A]{fontenc}
\usepackage[russian]{babel}
\usepackage[]{float}

\usepackage[oglav, boldsect, eqwhole, figwhole, %
   remarks, hyperref, hyperprint]{fn2kursstyle}

\frenchspacing
\righthyphenmin=2

%Командна для римских прописных 
\newcommand{\RomanNumeralCaps}[1]
    {\MakeUppercase{\romannumeral #1}}

\title{Прямые методы решения систем линейных алгебраических уравнений}
\group{ФН2-52Б}
\author{Ю.\,А.~Сафронов}
\supervisor{}
\date{2021}

\begin{document}
\maketitle
\tableofcontents 
\newpage

\section{Краткое описание алгоритмов}
Дана система линейных алгебраических уравнений:
\begin{equation}
\sum_{j=1}^{n} a_{ij}x_i = f_i , \quad i = \overline{1,n}.
\label{Sys}
\end{equation}

\subsection{Метод Гаусса}
Сначала система (\ref{Sys}) приводится прямым ходом к верхнетреугольному виду: 
\[\left\{
\begin{aligned}
a_{11}^{(0)}x_1 + a_{12}^{(0)}x_2 + a_{13}^{(0)}x_3 + ... + a_{1n}^{(0)}x_n = f_1^{(0)},\\
a_{22}^{(1)}x_2 + a_{23}^{(1)}x_3 + ... + a_{2n}^{(1)}x_n = f_2^{(1)},\\
...........................................................\\
a_{n-1,n-1}^{(n-2)}x_{n-1} + a_{n-1,n}^{(n-2)}x_n = f_{n-1}^{(n-2)},\\
a_{nn}^{(n-1)}x_n = f_{n}^{(n-1)}.\\
\end{aligned}
\right.
\]
Коэффициенты $a_{ij}^{(k)}$ и $f_i^{(k)}$ вычисляются следующим образом
\[
a_{ij}^{(k)} = a_{ij}^{(k-1)} - c_{ik}a_{kj}^{(k-1)}, \quad f_{i}^{(k)} = f_{i}^{(k-1)} - c_{ik}f_{k}^{(k-1)}, 
\]
где
\[
c_{ik} = \dfrac{a_{ik}^{(k-1)}}{a_{kk}^{(k-1)}}, \quad a_{ij}^{(0)}=a_{ij}, \quad f_{i}^{(0)} = f_i, \quad k = \overline{1,n-1},\quad j =\overline{k,n}, \quad i = \overline{k+1,n}.
\]

Далее производится обратный ход метода, во время которого определяются неизвестные $x_i$, начиная с $i = n$:
\[
x_i =\left(f_i^{(i-1)}-\sum_{j=i+1}^{n} a_{ij}^{(i-1)}x_j\right)/a_{ii}^{(i-1)}, \quad i = \overline{n,1}.
\]
Общее количество делений и умножений в методе Гаусса: $\dfrac{1}{3}n(n^2+3n-1) \sim \dfrac{n^3}{3}$.
\newpage

\section{Исходные данные}
\newpage

\section{Результаты расчетов}
\newpage

\section{Анализ результатов}
\newpage

\section{Контрольные вопросы}
\begin{enumerate}
 \item {\bf Каковы условия применимости метода Гаусса без выбора
и с выбором ведущего элемента?}

Метод Гаусса применим тогда и только тогда, когда все угловые миноры матрицы $\mathcal{A}$ ненулевые, что равносильно условию ${a_{ii}}^{(i-1)} \ne 0$ для всех $i = 1,2 , ..., n$, где ${a_{ii}}^{(i-1)}$ - элементы матрицы на главной диагонали после приведения ее к ступенчатому виду. Соотвественно, в противном случае метод Гаусса без выбора главного элемента в ходе работы может привести к делению на ноль, при этом матрица может быть и невырождена. Метод Гаусса с выбором главного элемента можно применять для любой невырожденной матрицы. Если матрица будет вырожденной, то в какой-то момент главный элемент будет равен нулю, что недопустимо. 

 \item {\bf Докажите, что если $\det \mathcal{A} \ne 0$, то при выборе главного
элемента в столбце среди элементов, лежащих не выше
главной диагонали, всегда найдется хотя бы один элемент,
отличный от нуля.}

Докажем от противного. Допустим, что возможна такая ситуация, когда при условии $\det \mathcal{A} \ne 0$, существует такой шаг $k$, для которого, соотвественно, в $k$-ом столбце все элементы не выше главной диагонали нулевые (на примере матрицы $n\times n$):

\[  \mathcal{A} = 
\begin{pmatrix}
a_{11} & a_{12} & ...& a_{1,k-1}&a_{1k}&...& a_{1, n-1} & a_{1n}\\
0 & a_{22} & ...& a_{2,k-1}&a_{2k} &...& a_{2, n-1} & a_{2n}\\\
... & ... & ...& ...& ... & ... & ... & ...\\
0 & 0 & ...& a_{k-1, k-1} &a_{k-1, k} & ... & a_{k, n-1}& a_{k n}\\
0 & 0 & ...& 0 &0 & ... & a_{k +1, n-1}& a_{k+1, n}\\
... & ... & ...& ... & ... & ... & ...&...\\
0 & 0 & ...& 0&0 & ... & a_{n-1, n-1} & a_{n-1, n}\\
0 & 0 & ...&0&0 & ... & 0 & a_{nn}\\
\end{pmatrix}.
\]

Определитель ступенчатой матрицы равен произведению элементов ее главной диагонали:

\[
\det \mathcal{A} = a_{11} * a_{22} * ... * a_{k-1, k-1} * 0 * a_{k+1, k+1} * ... * a_{nn}, \quad a_{kk} = 0.
\]

Противречие. Следовательно, либо матрица вырождена, либо существует ненулевой элемент не выше главной диагонали. 

\item{\bf В методе Гаусса с полным выбором ведущего элемента
приходится не только переставлять уравнения, но и менять нумерацию неизвестных. Предложите алгоритм, позволяющий восстановить первоначальный порядок неизвестных.}

Данную проблему можно решить вводом косвенной индексации. Вместо $\mathcal{A}[i][j]$ использовать $\mathcal{A}[row(i)][col(j)]$, где $row$ и $col$ --- массивы (по сути своей являющиеся подстановками), в которых, например, для перемены местами двух строк или столбцов нужно поменять местами соотвествующие индексы.
\end{enumerate}
\newpage




\end{document} 
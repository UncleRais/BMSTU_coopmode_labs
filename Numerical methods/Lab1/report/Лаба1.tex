\documentclass[12pt, a4paper]{article}

\usepackage[utf8]{inputenc}
\usepackage[T2A]{fontenc}
\usepackage[russian]{babel}
\usepackage[]{float}

\usepackage[oglav, boldsect, eqwhole, figwhole, %
   remarks, hyperref, hyperprint]{fn2kursstyle}

\frenchspacing
\righthyphenmin=2

%Командна для римских прописных 
\newcommand{\RomanNumeralCaps}[1]
    {\MakeUppercase{\romannumeral #1}}

\title{Прямые методы решения систем линейных алгебраических уравнений}
\group{ФН2-52Б}
\author{Ю.\,А.~Сафронов}
\supervisor{}
\date{2021}

\begin{document}
\maketitle
\tableofcontents 
\newpage

\section{Краткое описание алгоритмов}
\newpage

\section{Исходные данные}
\newpage

\section{Результаты расчетов}
\newpage

\section{Анализ результатов}
\newpage

\section{Контрольные вопросы}
\begin{enumerate}
 \item {\bf Каковы условия применимости метода Гаусса без выбора
и с выбором ведущего элемента?}

Метод Гаусса применим тогда и только тогда, когда все угловые миноры матрицы $\mathcal{A}$ ненулевые, что равносильно условию ${a_{ii}}^{(i-1)} \ne 0$ для всех $i = 1,2 , ..., n$, где ${a_{ii}}^{(i-1)}$ - элементы матрицы на главной диагонали после приведения ее к ступенчатому виду. Соотвественно, в противном случае метод Гаусса без выбора главного элемента в ходе работы может привести к делению на ноль, при этом матрица может быть и невырождена. Метод Гаусса с выбором главного элемента можно применять для любой невырожденной матрицы. Если матрица будет вырожденной, то в какой-то момент главный элемент будет равен нулю, что недопустимо. 

 \item {\bf Докажите, что если $\det \mathcal{A} \ne 0$, то при выборе главного
элемента в столбце среди элементов, лежащих не выше
главной диагонали, всегда найдется хотя бы один элемент,
отличный от нуля.}

Докажем от противного. Допустим, что возможна такая ситуация, когда при условии $\det \mathcal{A} \ne 0$, существует такой шаг $k$, для которого, соотвественно, в $k$-ом столбце все элементы не выше главной диагонали нулевые (на примере матрицы $n\times n$):

\[
\begin{pmatrix}
a_{11} & a_{12} & ...&a_{k1}&...& a_{1 n-1} & a_{1n}\\
0 & a_{22} & ...&a_{k2} &...& a_{2 n-1} & a_{2n}\\\
... & ... & ...& ... & ... & ... & ...\\
0 & 0 & a_{k-1 k-1} &a_{k, k - 1} & ... & a_{k n-1}& a_{k n}\\
0 & 0 & 0 &0 & ... & a_{k +1 n-1}& a_{k+1 n}\\
... & ... & ...& ... & ... & ... & ...\\
0 & 0 & ...&0 & ... & a_{n-1 n-1} & a_{n-1 n}\\
0 & 0 & ...&0 & ... & 0 & a_{nn}\\
\end{pmatrix}
\]

\end{enumerate}
\newpage




\end{document} 
\documentclass[12pt, a4paper]{article}

\usepackage[utf8]{inputenc}
\usepackage[T2A]{fontenc}
\usepackage[russian]{babel}
\usepackage[]{float}

\usepackage[oglav, boldsect, eqwhole, figwhole, %
   remarks, hyperref, hyperprint]{fn2kursstyle}

\frenchspacing
\righthyphenmin=2

%Командна для римских прописных 
\newcommand{\RomanNumeralCaps}[1]
    {\MakeUppercase{\romannumeral #1}}

\title{Прямые методы решения систем линейных алгебраических уравнений}
\group{ФН2-52Б}
\author{A.\,И.~Токарев}
\secauthor{Ю.\,А.~Сафронов}
\supervisor{}
\date{2021}

\def\hmath$#1${\texorpdfstring{{\rmfamily\textit{#1}}}{#1}}

\begin{document}
\maketitle
\tableofcontents 
\newpage

\section{Краткое описание алгоритмов}
Дана система линейных алгебраических уравнений:
\begin{equation}
\sum_{j=1}^{n} a_{ij}x_i = f_i , \quad i = \overline{1,n}.
\label{Sys}
\end{equation}



\section{Исходные данные}
Даны две СЛАУ, которые имеют вид: 

\[
    A = 
    \begin{pmatrix}
     28.8590    &    -0.0080    &     2.4060    &    19.2400    \\   
     14.4360    &    -0.0010    &     1.2030    &     9.6240    \\   
    120.2040    &    -0.0320    &    10.0240    &    80.1440    \\  
    -57.7140    &     0.0160    &    -4.8120    &   -38.4780       

    \end{pmatrix}, \quad f_{A} = 
    \begin{pmatrix}
    30.4590  \\
    18.2480  \\
    128.1560 \\
    -60.9080 \\
    \end{pmatrix},
\]

\[
    B = 
    \begin{pmatrix}
        117.2000 &    1.0500   &    -8.9700 &    0.7500 \\
        4.2600  &    185.8000   &   0.1300 &   -8.8600  \\
        -3.8100  &    5.2300  &  -189.0000  &  -4.8800  \\
        5.8200  &      3.8700  &   -2.4700   &    81.4000   
    \end{pmatrix}, \quad f_{B} = 
    \begin{pmatrix}
       455.3400 \\
       -924.0400  \\
      -1554.4600  \\
        59.7500
    \end{pmatrix}
\]

\newpage

\section{Результаты расчетов}
Результаты для A:
\begin{enumerate}
\item Точность double
	\begin{enumerate}
	\item[a)] Метод Гаусса
	$$x^* = (1{.}000 , 1000{.}000, -20{.}000 ,3{.}000)^{T},\quad ||Ax^{*}-b|| = 5{.}75\cdot10^{-14}.$$
	\item[б)] Метод QR
	$$x^* = (1{.}000 , 1000{.}000, -20{.}000 ,3{.}000)^{T},\quad ||Ax^{*}-b|| = 9{.}11\cdot10^{-14}.$$
	\end{enumerate}
\item Точность float
	\begin{enumerate}
	\item[a)] Метод Гаусса
	$$x^* = (1{.}487 , 1000{.}238, -18{.}078 ,2{.}029)^{T},\quad ||Ax^{*}-b|| = 3{.}303\cdot10^{-5}.$$
	\item[б)] Метод QR
$$x^* = (1{.}313 , 1000{.}154, -18{.}766 ,2{.}377)^{T},\quad ||Ax^{*}-b|| = 7{.}864\cdot10^{-6}.$$
	\end{enumerate}
\end{enumerate}
Изменим вектор $b$ на величину $\delta = 0.01$. Тогда для точности double методом Гаусса
$$b^* = (30.4690, 18{.}2580, 128{.}1660, -60{.}9180)^T,$$
$$x^* = (-1278{.}8167, 378{.}425, -5019{.}792, 2547{.}633), \quad ||Ax^{*}-b^*|| =  5{.}15\cdot10^{-11}.$$

Для точности float методом Гаусса
$$b^* = (30.4690, 18{.}2580, 128{.}1660, -60{.}9180)^T,$$
$$x^* = (-1006{.}303, 513{.}317, -3939{.}908, 2003{.}767), \quad ||Ax^{*}-b^*|| =  0{.}016.$$

Малое изменение правой части ведет к большому изменению решения, следовательно, матрица плохо обусловлена. Точный расчет числа обусловленности: 
$$cond_1A = 1{.}22\cdot10^8,\quad cond_{\infty}A = 1{.}09\cdot10^8, \quad cond_{max}A = 5{.}63\cdot10^8 .$$ 
Оценка числа обусловленности снизу:
$$cond_A = 42319{.}177.$$

Результаты для B:
\begin{enumerate}
\item Точность double
	\begin{enumerate}
	\item[a)] Метод Гаусса
	$$x^* = (3{.}000 , -5{.}000, 8{.}000 ,1{.}000)^{T},\quad ||Ax^{*}-b|| = 2{.}163\cdot10^{-12}.$$
	\item[б)] Метод QR
	$$x^* = (2{.}999 , -5{.}000, 8{.}000 ,0{.}999)^{T},\quad ||Ax^{*}-b|| = 3{.}019\cdot10^{-11}.$$
	\end{enumerate}
\item Точность float
	\begin{enumerate}
	\item[a)] Метод Гаусса
	$$x^* = (3{.}000 , -4{.}999, 8{.}000, 1{.}000)^{T},\quad ||Ax^{*}-b|| = 0{.}001.$$
	\item[б)] Метод QR
	$$x^* = (3{.}000 , -5{.}000, 8{.}000, 0{.}999)^{T},\quad ||Ax^{*}-b|| = 0{.}010.$$
	\end{enumerate}
\end{enumerate}
Изменим вектор $b$ на величину $\delta = 0.01$. Тогда для точности double методом Гаусса
$$b^* = (455.3500, -924{.}0500, -1554{.}4700, 59{.}7500)^T,$$
$$x^* = (2{.}999, -5{.}000, 8{.}000, 0{.}999), \quad ||Ax^{*}-b^*|| =  2{.}24\cdot10^{-12}.$$

Для точности float методом Гаусса
$$b^* = (455.3500, -924{.}0500, -1554{.}4700, 59{.}7500)^T,$$
$$x^* = (2{.}999, -5{.}000, 8{.}000, 0{.}999), \quad ||Ax^{*}-b^*|| =  0{.}001.$$

Малое изменение правой части ведет к малому изменению решения, следовательно, матрица хорошо обусловлена. Точный расчет числа обусловленности: 
$$cond_1A = 2{.}64,\quad cond_{\infty}A = 2{.}64, \quad cond_{max}A = 37{.}05.$$
Оценка числа обусловленности снизу:
$$cond_A = 1{.}40$$
\newpage

\section{Анализ результатов}


\newpage

\section{Контрольные вопросы}
\begin{enumerate}

\item s
\end{enumerate}
\newpage




\end{document} 
\documentclass[12pt, a4paper]{article}

\usepackage[utf8]{inputenc}
\usepackage[T2A]{fontenc}
\usepackage[russian]{babel}
\usepackage[]{float}

\usepackage[oglav, boldsect, eqwhole, figwhole, %
   remarks, hyperref, hyperprint]{fn2kursstyle}

\frenchspacing
\righthyphenmin=2

%Командна для римских прописных 
\newcommand{\RomanNumeralCaps}[1]
    {\MakeUppercase{\romannumeral #1}}

\title{Итерационные методы решения систем линейных алгебраических уравнений}
\group{ФН2-52Б}
\author{A.\,И.~Токарев}
\secauthor{Ю.\,А.~Сафронов}
\supervisor{}
\date{2021}

\def\hmath$#1${\texorpdfstring{{\rmfamily\textit{#1}}}{#1}}

\begin{document}
\maketitle
\tableofcontents 
\newpage

\section{Краткое описание алгоритмов}
Дана система линейных алгебраических уравнений:
\begin{equation}
\sum_{j=1}^{n} a_{ij}x_i = f_i , \quad i = \overline{1,n}.
\label{Sys}
\end{equation}

Будем искать решение итерационными методами, т.е. последовательно приближаясь к решению. Общий вид стационарных итерационных методов:

\[
B\dfrac{x^{k+1}-x^k}{\tau} + Ax^k = f.
\]

\subsection{Метод простой итерации}

\[
B = E, \quad \dfrac{x^{k+1}-x^k}{\tau} + Ax^k = f, \quad k \in \mathbb{N};
\]
\[
x^k = -(\tau A - E)x^{k-1} + \tau f.
\]

\subsection{Метод Якоби}
Представим матрицу $A$ в виде суммы
\[
A = L + D + U,
\]
где $L$ --- нижняя треугольная, $U$ --- верхняя треугольная, а $D$ - диагональная матрицы.

\[
B = D, \quad D(x^{k-1}-x^k) +  Ax^k = f, \quad k \in \mathbb{N}; 
\]
\[
 x^{k+1} = -D^{-1}(L + U)x^k + D^{-1}f.
\]

\subsection{Методы релаксации и Зейделя}

\[
B = D + \omega L, \quad \tau = \omega \quad (D + \omega L)\dfrac{x^{k+1} - x^k}{\omega} + Ax^k = f, \quad k \in \mathbb{N};
\]
\[
(E + \omega D^{-1}L)x^{k+1} = ((1 - \omega)E - \omega D^{-1}U)x^k + \omega D^{-1}f;
\]

$\omega > 0$ --- метод релаксации, $\omega = 1$ --- частный случай, метод Зейделя.

Для метода релаксации существуют удобные расчетные формулы, которые помогают упростить вычисления. 

\[
x_{i}^{k+1} + \omega\sum_{j = 1}^{i - 1}\dfrac{a_{ij}}{a_{ii}}x_{j}^{k+1} = (1 - \omega)x_i^k - \omega\sum_{j = i + 1}^{n}\dfrac{a_{ij}}{a_{ii}}x_j^k + \omega\dfrac{f_i}{a_{ii}}, i=\overline{1,..., n} ;
\]
\[
x_1^{k+1} = (1 - \omega)x_1^k - \omega\sum_{j = 2}^{n}\dfrac{a_{1j}}{a_{11}}x_j^k + \omega \dfrac{f_1}{a_{11}};
\]

\[
x_2^{k+1} = -\omega\dfrac{a_{21}}{a_{22}}x_1^{k+1} +(1 - \omega)x_2^k - \omega\sum_{j = 3}^{n}\dfrac{a_{2j}}{a_{22}}x_j^k + \omega \dfrac{f_2}{a_{22}};
\]

\centerline{...}

\[
x_i^{k+1} = -\omega\sum_{j=1}^{i-1}\dfrac{a_{ij}}{a_{ii}}x_j^{k+1} +(1 - \omega)x_i^k - \omega\sum_{j = i+1}^{n}\dfrac{a_{ij}}{a_{ii}}x_j^k + \omega \dfrac{f_i}{a_{ii}}, \quad i = \overline{3,4,..., n-1};
\]

\[
x_n^{k+1} = -\omega\sum_{j=1}^{i-1}\dfrac{a_{nj}}{a_{nn}}x_j^{k+1} +(1 - \omega)x_n^k  + \omega \dfrac{f_n}{a_{nn}}
\]




\section{Исходные данные}
Для СЛАУ матрица $A$ и столбец правой части $f_A$ имеют вид 

\[
    A = 
    \begin{pmatrix}
     175.4000    &     0.0000   &      9.3500    &    -0.960\\
  0.5300  &     -46.0000   &      0.2300    &     5.1900\\
 -0.6300    &     5.4400  &     190.6000     &    9.7000\\
  6.2300     &   -8.8900     &   -9.8800    &  -153.4000
  

    \end{pmatrix}, \quad f_{A} = 
    \begin{pmatrix}
 985.3600\\
 348.170\\
2284.7700\\
-638.7800\\

    \end{pmatrix}.
\]

Так же нужно решить СЛАУ с трехдиагональной матрицей $A$ размерности $n = 220$. 

\[
   \begin{pmatrix}
     b_1   &     c_1   &      0    &    ... & 0\\
     a_2 &     b_2   &      c_2    &   ... & 0\\
     0   &    a_3  &     b_3     &   ... & 0 \\
     ...   &    ...  &    ...     &   ... & ...\\ 
     0 & ...    &   a_{n-1}    &   b_{n-1}   &  c_{n-1} \\
     0 & ... & 0 & a_n & b_n\\
    \end{pmatrix} \cdot
    \begin{pmatrix}
 x_1\\
 x_2\\
x_3\\
...\\
x_{n-1}\\
x_n
    \end{pmatrix} =
\begin{pmatrix}
 f_1\\
 f_2\\
f_3\\
...\\
f_{n-1}\\
f_n
    \end{pmatrix},
\]
$a_i = c_i = 1; b_i = 4; i=\overline{1,...,n}; d_1 = 6; d_i = 10 - 2(i \mod 2), i = \overline{2,..., n-1}; d_n = 9 - 3(n \mod 2).$
\newpage

\section{Результаты расчетов}
\noindent\begin{center}
\begin{tabular}{|c|c|c|c|c|c|c|c|c|}
\hline
Метод & $||C||$ или  & Оценка  & Крит. 1.  & Крит. 1. & Крит. 2.  &Крит. 2. & Крит. 3.& Крит. 3.  \\
&$||G_1||$ & числа &Норма&  Число  & Норма & Число &  Норма& Число  \\
& $+ ||G_2||$ &итераций & ошибки &  итераций & ошибки & итераций &ошибки & итераций \\
\hline
&&&&&&&& \\
\hline
&&&&&&&&\\
\hline
&&&&&&&& \\
\hline
&&&&&&&& \\
\hline
&&&&&&&& \\
\hline
&&&&&&&& \\
\hline
\end{tabular}
\end{center}
\newpage

\section{Анализ результатов}
\newpage

\section{Контрольные вопросы}
\begin{enumerate}
\item Почему условие $||C|| < 1$ гарантирует сходимость итерационных методов решения СЛАУ?


\end{enumerate}
\newpage




\end{document} 
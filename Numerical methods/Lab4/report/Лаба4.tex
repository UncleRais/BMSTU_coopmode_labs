\documentclass[12pt, a4paper]{article}

\usepackage[utf8]{inputenc}
\usepackage[T2A]{fontenc}
\usepackage[russian]{babel}
\usepackage[]{float}

\usepackage[oglav, boldsect, eqwhole, figwhole, %
   remarks, hyperref, hyperprint]{fn2kursstyle}

\frenchspacing
\righthyphenmin=2

%Командна для римских прописных 
\newcommand{\RomanNumeralCaps}[1]
    {\MakeUppercase{\romannumeral #1}}

\title{Методы решения проблемы собственных значений}
\group{ФН2-52Б}
\author{A.\,И.~Токарев}
\secauthor{Ю.\,А.~Сафронов}
\supervisor{}
\date{2021}

\def\hmath$#1${\texorpdfstring{{\rmfamily\textit{#1}}}{#1}}

\begin{document}
\maketitle
\tableofcontents 
\newpage

\section{Краткое описание алгоритмов}


\section{Исходные данные}

\newpage

\section{Результаты расчетов}


\section{Контрольные вопросы}
\begin{enumerate}
\item Почему нельзя находить собственные числа матрицы A, прямо решая уравнение $det(A - \lambda E) = 0$, а собственные векторы --- <<по определению>>, решая систему $(A - \lambda_i E)e_i = 0$?

\item Докажите, что ортогональное преобразование подобия сохраняет симметрию матрицы.

\item Как преобразование подобия меняет собственные векторы
матрицы?

Можно рассматривать матрицу P как матрицу перехода. $B = P^{-1}AP$ матрицы $A$ и $B$ подобны. 


\item Почему на практике матрицу A подобными преобразованиями вращения приводят только к форме Хессенберга, но не к треугольному виду?

\item Оцените количество арифметических операций, необходимое для приведения произвольной квадратной матрицы A к форме Хессенберга.

\item Сойдется ли алгоритм обратных итераций, если в качестве начального приближения взять собственный вектор, соответствующий другому собственному значению? Что будет в этой ситуации в методе обратной итерации, использующем отношение Рэлея?

\item Сформулируйте и обоснуйте критерий останова для $QR$-алгоритма отыскания собственных значений матрицы.

\item Предложите возможные варианты условий перехода к алгоритму со сдвигами. Предложите алгоритм выбора величины сдвига.

\item Для чего нужно на каждой итерации нормировать приближение к собственному вектору?

\item Приведите примеры использования собственных чисел и собственных векторов в численных методах.
\end{enumerate}

\end{document} 
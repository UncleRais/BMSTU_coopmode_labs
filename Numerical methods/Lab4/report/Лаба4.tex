\documentclass[12pt, a4paper]{article}

\usepackage[utf8]{inputenc}
\usepackage[T2A]{fontenc}
\usepackage[russian]{babel}
\usepackage[]{float}

\usepackage[oglav, boldsect, eqwhole, figwhole, %
   remarks, hyperref, hyperprint]{fn2kursstyle}

\frenchspacing
\righthyphenmin=2

%Командна для римских прописных 
\newcommand{\RomanNumeralCaps}[1]
    {\MakeUppercase{\romannumeral #1}}

\title{Методы решения проблемы собственных значений}
\group{ФН2-52Б}
\author{A.\,И.~Токарев}
\secauthor{Ю.\,А.~Сафронов}
\supervisor{}
\date{2021}

\def\hmath$#1${\texorpdfstring{{\rmfamily\textit{#1}}}{#1}}

\begin{document}
    \maketitle
    \tableofcontents 
    \newpage

    \section{Краткое описание алгоритмов}

    \subsection{Метод $QR$ разложения}

    Один из способов нахождения собственных значений квадратной матрицы --- приведение данной матрицы к треугольному виду преобразованием подобия
    \[
        R=P^{-1}AP,
    \]

    \noindent где $P-$невырожденная матрица, которую можно найти, используя $QR-$алгоритм. Рассмотрим метод $QR-$разложения или алгоритмом Френсиса-Кублановской.

    На первой итерации строится $QR-$разложение матрицы $A^{(0)}=A\colon$
    \[
        A^{(0)}=Q_{1}R_{1},~~~ следовательно R_{1}=Q_{1}^{-1}A^{(0)}.
    \]

    Затем вычислим матрицу $A^{(1)}=R_{1}Q_{1}$ или $A^{(1)}=Q_{1}^{-1}A^{(0)}Q_{1}$.\\
    Видим, что $A^{(0)}$ и $A^{(1)}$ подобны и имеют один и тот же набор собственных значений.\\
    На второй итерации найдём $QR-$разложение матрицы $A^{(1)}$ и вычисляется $A^{(2)}.$ На $(k+1)-$ой итерации определим разложение $A^{(k)}=Q_{k+1}R_{k+1}$ и построим матрицу 
    \[
        A^{(k+1)}=R_{(k+1)}Q_{(k+1)}=Q_{(k+1)}^{-1}A^{(k)}Q_{k+1}.
    \]
    
    Получим последовательность матриц $\{A^{(k)}\},$в том случае, если собственные значения $A$ вещественны и различны по модулю, т.е. $|\lambda_{1}|>|\lambda_{2}|>\ldots > |\lambda_{n}|,$ сходится к верхнетреугольной матрице. Отметим, что элементы $a_{ij}^{(k)}$ матриц $A^{(k)}$, стоящие ниже главной диагонали, сходятся к нулю со скоростью геом-ой прогрессии, т.е.:
    \[
        |a_{ij}^{(k)}| \leq |\frac{\lambda_{i}}{\lambda_{j}}| \cdot |a_{ij}^{(k-1)}|,~~~i>j,~k=1,2,\ldots. 
    \]

    Отметим, что среди С.Ч матрицы $A$ есть близкие величины, то есть
    \[
        |\frac{\lambda_{i}}{\lambda_{j}}| \approx 1,
    \]
    то сходимость будет очень медленной. Поэтому используют алгоритм со сдвигами: \\
    ищем собственные значения матрицы $\widetilde{A} = A - \sigma E,$ которые равны $\widetilde{\lambda_{i}} = \lambda_{i} - \sigma.$
    В таком случае скорость сходимости $QR-$алгоритма определяется величиной
    \[
        |\frac{\widetilde{\lambda_{i}}}{\widetilde{\lambda_{j}}}|=|\frac{\lambda_{i}-\sigma}{\lambda_{j}-\sigma}|.
    \]

    \subsection{Метод обратных итераций}

    Если известно собственное значение $\lambda_{i}$ матрицы $A$ или точное приближение $\lambda_{i}^{*},$ то можно рассмотреть задачу нахождения ссобственного вектора, отвечающему данному собственному значению.

    Собственный вектор $e_{i}$ ищем как нетривиальное решение системы линейных алгебраических уравнений
    \[
        (A-\lambda_{i}E)e_{i}=0
    \]
    с вырожденной матрицей $(A-\lambda_{i}E)$. Когда $\lambda_{i}$ известно приближенно, тогда нужно решать систему
    \[
        (A-\lambda_{i}^{*}E)e_{i},
    \]
    решение которой может быть только тривиальным, так как матрица $(A-\lambda_{i}^{*}E)$ невырождена. Поэтому численное решение данной системы не даёт возможности вычислить соответствующий собственный вектор.

    Тогда рассмотрим \textbf{метод простых итераций}. Каждая итерация данного метода состоит из двух этапов:

    \begin{enumerate}
        \item На первои этапе решается система 
        \[
            (A-\lambda_{i}^{*}E)y^{(k+1)}=x^{(k)}
        \]
        относительно неизвестного вектора $y^{k+1}$.

        \item На втором этапе производится нормировка решения:
        \[
            x^{(k+1)}=\frac{y^{(k+1)}}{\|y^{(k+1)}\|}. 
        \]
        В качестве $x^{(0)}$ можно взять любой нормированный вектор. При условии, что известное приближение $\lambda_{i}^{*}$ достаточно близко к истинному значению $\lambda_{i}$, последовательность векторов $x^{k}$ быстро сходится к собственному вектору $e_{i}$, соотвествующему собственному значению $\lambda_{i}$.
    \end{enumerate}

    \newpage

    \section{Исходные данные}

    Даны матрицы ($dim = 4$):
    \[
        A_{20} = 
        \begin{pmatrix}
            99.4000  &   -2.9000  &   -9.9800   &   0.6300  \\  
            -2.9000  &  106.4000  &   -9.4300   &  -8.0200  \\  
            -9.9800  &   -9.4300  & -159.4000   &  -5.8900  \\  
             0.6300  &   -8.0200  &   -5.8900   &  58.2000  
        \end{pmatrix}
    \]
    \[
        A_{23} = 
        \begin{pmatrix}
            -182.2000  &    1.6100   &   7.6500  &   -9.1000  \\
               1.6100  &  -43.4000   &   5.5400  &   -6.8500  \\
               7.6500  &    5.5400   &  12.6000  &    9.3200  \\
              -9.1000  &   -6.8500   &   9.3200  &  -77.2000  \\
        \end{pmatrix}
    \]
    \newpage

    \section{Результаты расчетов}


    \section{Контрольные вопросы}
    \begin{enumerate}
    \item {\bf Почему нельзя находить собственные числа матрицы A, прямо решая уравнение $det(A - \lambda E) = 0$, а собственные векторы --- <<по определению>>, решая систему $(A - \lambda_i E)e_i = 0$? }
    
    Для нахождения собственных чисел матрица, нам надо составить характеристический многочлен и найти его корни, что для многочленов высокой степени является трудоемким процессом. Данный подход становится неудовлетворительным, если речь идёт о вычислении собственных значений матриц, имающих порядок $m$ в несколько десятков ( или даже сотен). Одна из причин состоит в том, что $Ax=\lambda x$ и $\lambda^{m}+p_{1}\lambda^{m-1}+p_{2}\lambda^{m-2}+\ldots+p_{m-1}\lambda + p_{m}=0$ формально эквивалентны, они имеют разную обусловленность. Так как корни многочлена $P_{m}(x)$ высокой степени срезвычайно чувствительны к погрешностям в коэффициентах, то на этапе вычисления коэффициентов характеристического уравнения может быть в значительной степени потеряна информация о собственнных значениях матрицы. Если исходить непосредственно из определения собственного вектора, то $ e_{i} $ следует искать как нетривиальное решение системы линейных алгебраических уравнений 
	\[ (A - \lambda_{i} E) e_{i} = 0 \]
	с вырожденной матрицей $ (A - \lambda_{i} E) $. Но обычно $ \lambda_{i} $ известно лишь приближенно, и в действительности приходится решать систему 
	\[ (A - \lambda_{i}^{\ast} E) e_{i} = 0, \]
	где $ \lambda_{i}^{\ast} $ --- достаточно точное приближение к собственному значению $ \lambda_{i} $. Решение данной системы быть только тривиальным, так как матрица $ (A - \lambda_{i}^{\ast} E) $ невырождена. Поэтому непосредственное численное решение не дает возможности вычислить соответствующий собственный вектор.
    
    \newpage

    \item {\bf Докажите, что ортогональное преобразование подобия сохраняет симметрию матрицы. }
    
    Ортогональное преобразование подобия имеет вид:
	\[ R = P^{-1} A \, P, \]
	где $ P^{-1} = P^{T}, \: A = A^{T} $. Тогда 
	\begin{gather*}
		R^{T} = \left( P^{-1} A \, P \right)^{T} = \left( A \, P \right)^{T}\left(P^{-1}\right)^{T}  = P^{T} A^{T} \left( P^{-1} \right)^{T} = P^{-1} A \, P = R, \\
		\Longrightarrow R^{T} = R.
	\end{gather*}


    \item {\bf Как преобразование подобия меняет собственные векторы }
    матрицы?

    Можно рассматривать матрицу P как матрицу перехода. $B = P^{-1}AP$ матрицы $A$ и $B$ подобны. 
    
    Полученная в результате преобразования подобия матрица имеет тот же набор собственных чисел:
	\begin{multline*}
		\det \left( P^{-1} A \, P - \lambda E \right) = \det \left( P^{-1} (A - \lambda E) \, P \right) = \\
		= \det \left( P^{-1} \right) \det \left( A - \lambda E \right) \det \left( P \right) = \det \left( A - \lambda E \right).
	\end{multline*}
	Таким образом, характеристические многочлены и собственные числа матриц $ A $ и $ P^{-1} A \, P $ совпадают. Соответствующие собственные векторы $ x $ и $ x^{'} $ не совпадают, но, т.к $ P^{-1} A \, P \, x^{'} = \lambda \, x^{'} \Rightarrow A \, P \, x^{'} = \lambda \, P^{-1} x^{'} $, они связаны равенством $ x = P \, x^{'} $.


    \item {\bf Почему на практике матрицу A подобными преобразованиями вращения приводят только к форме Хессенберга, но не к треугольному виду? }
    
    Рассматривая алгоритм подобных преобразований вращения, заметим, что обнуляются все элементы, лежащие левее элемеента стоящего на поддиагонали, то построив такую последовательность элементарных вращений, которая приведет матрицу $A$ к форме Хессенберга:
    \[
        A^*=T_{kl}AT_{kl}^{-1}=T_{kl}AT_{kl}^T
    \]

    где $T_{kl}$-матрица, в которой все элементы главной диагонали равны 1, кроме элементов стоящих на пересечении $k$-ых столбца и строки и $l$-ых столбца и строки(они равны $\alpha = \cos \phi$), а все элементы вне главной диагонали равны 0, за исключением элемента на пересечении $k$-ого столбца и $l$-ой строки(он равен $-\beta=-\sin\phi$) и элемента стоящего на пересечении $l$-ого столбца и $k$-ой строки(он равен $\beta=\sin\phi$), а $A^*$ отличается от матрицы $A$ лишь двумя строками и двумя столбцами с номерами $k, l$, при этом в матрице $A^*$ элемент $a_{l,k-1}^*=0$.

    По построению матриц $T_{kl}$ видно, что $k>l$ и $k>1$, а значит можем сделать вывод, что не возможно данными преобразованиями занулить поддиагональные элементы.

    \item {\bf Оцените количество арифметических операций, необходимое для приведения произвольной квадратной матрицы A к форме Хессенберга. }
    
    Для вычисления элементов матрицы $ T_{kl} $ требуется 5 мультипликативных операций. Необходимо обнулить все элементы ниже диагонали, примыкающей к главной в столбцах с 1 по $ n - 2 $.В $k$-ом столбце необходимо обнулить $ n - k - 1 $ элемент. Умножение слева и справа на матрицы $ T_{kl} $ и $ T_{kl}^{T} $ соответственно изменяет в матрице $ A $ $4n - 2k + 2 $ элемента. Для изменения одного элемента требуется 2 мультипликативные операции. В итоге получаем:
	\[ \sum \limits_{k = 1}^{n - 2} 5 \cdot (n - k - 1) (4 n - 2k + 2) \cdot 2 = \frac{50 n^3}{3} - 40 n^2 + \frac{10 n}{3} + 20. \]


    \item {\bf Сойдется ли алгоритм обратных итераций, если в качестве начального приближения взять собственный вектор, соответствующий другому собственному значению? Что будет в этой ситуации в методе обратной итерации, использующем отношение Рэлея? }
    
    В качестве начального приближения в методе обратных итераций можно взять любой нормированный вектор. Пусть $e_{i}, \; i = \overline{1,n}$ --- ОНБ из собственных векторов матрицы $A$.
	\begin{gather*}
		(A - {\lambda}_{j}^{\ast} E) y = x; \\
		y = \sum \limits_{i = 1}^{n} \alpha_{i} e_{i}, \; x = \sum \limits_{i = 1}^{n} c_{i} e_{i}; \\
		\sum \limits_{i = 1}^{n} \alpha_{i} (\lambda_{i} - {\lambda}_{j}^{\ast}) = \sum \limits_{i = 1}^{n} c_{i} e_{i}; \\	
		\alpha_{i} = \frac{c_{i}}{\lambda_{i} - {\lambda}_{j}^{\ast}}; \\
		y = \sum \limits_{i = 1}^{n} \frac{c_{i}}{\lambda_{i} - {\lambda}_{j}^{\ast}} e_{j} = \frac{1}{\lambda_{i} - {\lambda}_{j}^{\ast}} \left( c_{j} e_{j} + \sum \limits_{i \ne j} \frac{\lambda_{j} - {\lambda}_{j}^{\ast}}{\lambda_{i} - {\lambda}_{j}^{\ast}} c_{i} e_{i} \right).
	\end{gather*}
	
	Если в качестве начального приближения взять собственный вектор, соответствующий другому собственному числу:
	\[ y = \frac{1}{\lambda_{i} - {\lambda}_{j}^{\ast}} \left( c_{j} e_{j} + \frac{\lambda_{j} - {\lambda}_{j}^{\ast}}{\lambda_{k} - {\lambda}_{j}^{\ast}} c_{k} e_{k} \right). \]
	Если $ | \lambda_{j} - {\lambda}_{j}^{\ast} | \ll | \lambda_{k} - {\lambda}_{j}^{\ast} | $, то второе слагаемое правой части мало по сравнению с первым. Следовательно  алгоритм сойдется к $e_{j}$.
	
	Если в методе обратных итераций использовать отношение Рэлея, а в качестве начального приближения $ x^{(0)} $ выбрать собственный вектор $ e_{k} $, соответствующий другому собственному значению, то метод сойдется к собственному числу, соответствующему собственному вектору $ e_{k} $.

    \item {\bf Сформулируйте и обоснуйте критерий останова для $QR$-алгоритма отыскания собственных значений матрицы. }
    
    Так как последовательность матриц $A_k$ сходится к верхнетреугольной матрице $R$, на главной диагонали которой стоят собственные значения, то используя тот факт, что $QR$-алгоритм последовательно обнуляет элементы начиная с $a_{n,1}$ до $a_{n,n-1}$, то итерационный метод поиска собственного значения следует продолжать пока не будет выполняться неравество $|a_{n,n-1}|<\varepsilon$, затем считая что $\lambda_i=a_{n,n}$ переходить к задаче меньшей размерности т.е. искать спектр матрицы размерности $(n-1)\times(n-1).$

    \item {\bf Предложите возможные варианты условий перехода к алгоритму со сдвигами. Предложите алгоритм выбора величины  сдвига. }
    
    При помощи леммы Гершгорина можем оценить диапазон собственных значений и в случае, если оценка диапазона меньше единицы, то мы получим достаточное условие того, что отношение собственных значений близко к единице, и можно будет перейти к алгоритму со сдвигами, взяв в качестве величины сдвига среднее значение из оценки диапазона.\\
Так же если элементы $a_{ij}^{(k)}$ матриц $A^{(k)}$, стоящие ниже главной диагонали $\frac{a_{ij}^{(k)}}{a_{ij}^{(k-1)}}\leqslant\Bigl|\frac{\lambda_i}{\lambda_j}\Bigr|\approx 1$, $i>j$, $k=1,2,\dots$,\\ то алгоритм будет сходиться очень медленно и следует переходить к алгоритму со сдвигами. В качестве величины сдвига можно взять $a_{n,n}^{(k)}$.

    \item {\bf Для чего нужно на каждой итерации нормировать приближение к собственному вектору? }
    
    Если $ | \lambda | > 1 $, то последовательность норм векторов стремится к бесконечности, если $ | \lambda | < 1 $, то последовательность норм векторов стремится к нулю и возможно исчезновение порядка. Для предупреждения этих ситуаций вектор $ x^{k} $ нормируют.
То есть приближение к собственному вектору необходимо нормировать для того чтобы застраховаться от накопления погрешностей и выхода значений перенных за пределы типа.

    \item {\bf Приведите примеры использования собственных чисел и собственных векторов в численных методах. }
    
    \begin{enumerate}
		1) C помощью собственных чисел можно сделать вывод о числе обусловленности матрицы.\\
		2) В электрических и механических системах собственные числа отвечают собственным частотам колебаний,  а собственные векторы характеризуют соответствующие формы колебаний.\\
		3) Одна из задач, которая дает геометрическую интерпретацию собственных векторов, есть приведение кривых второго порядка к каноническому виду. Собственные вектора образуют главные направления кривых второго порядка.\\

	\end{enumerate}


    \end{enumerate}

\end{document} 
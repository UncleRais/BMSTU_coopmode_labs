\documentclass[12pt, a4paper]{article}

\usepackage[utf8]{inputenc}
\usepackage[T2A]{fontenc}
\usepackage[russian]{babel}
\usepackage[]{float}

\usepackage[oglav, boldsect, eqwhole, figwhole, %
   remarks, hyperref, hyperprint]{fn2kursstyle}

\frenchspacing
\righthyphenmin=2

%Командна для римских прописных 
\newcommand{\RomanNumeralCaps}[1]
    {\MakeUppercase{\romannumeral #1}}

\title{Решение задач интерполирования}
\group{ФН2-52Б}
\author{A.\,И.~Токарев}
\secauthor{Ю.\,А.~Сафронов}
\supervisor{}
\date{2021}

\def\hmath$#1${\texorpdfstring{{\rmfamily\textit{#1}}}{#1}}

\begin{document}
\maketitle
\tableofcontents 
\newpage

\section{Краткое описание алгоритмов}

\subsection{Равномерная сетка}

Шаг равномерной сетки постоянный и вычисляется по формуле: 

\[
    h = \dfrac{b - a}{n},  
\]

\noindent а сами узлы имеют координаты 

\[
    x_i = a + h \cdot i = a + \dfrac{b - a}{n} \cdot i, \quad i = 0,1,\ldots,n
\]

\subsection{Чебышевская сетка}

Узлы вычисляются, как корни многочлена Чебышева $1-$го рода, то есть точки 

\[
    x_i = \dfrac{a + b}{2} + \dfrac{b - a}{2} \cos \dfrac{(2i + 1) \pi }{2 (n + 1)}, \quad i = 0,1,\ldots,n
\]

\subsection{Задача интерполирования}

Задан отрезок $[a, b]$. Пусть точки $x_0 \ldots x_n$ –- узлы интерполяции, то есть точки, лежащие внутри этого отрезка. А значения $y(x_0) = y_0,\  \ldots \  ,y(x_n) = y_n$ -- значения искомой функции в этих точках. Послодовательность $\{y_i\}_{i=0}^n$ будем называть сеточной функцией. 

Таким образом, задача интреполирования заключается в построении такой функции $ f(x) $, которая будет принимать в узлах те же значения, что и $ y_i $. Геометрически это можно интерпретировать, как построение кривой, проходящей через систему точек $ (x_i, y_i) $

\subsection{Многочлен Лагранжа}

Многочлен $n$-степени вида
\[
    L_n(x) = \displaystyle \sum_{k = 0}^{n} \alpha_kx_k,  
\]

\noindent называют { \bf интерполяционным многочленом }, если 
\[
    L_n(x_i) = y_i   
\]

\pagebreak
{ \bf Интерполяционный многочлен Лагранжа }:

\[
    L_n(x) = \displaystyle \sum_{k = 0}^{n} c_k(x)y_k, \quad i = 0,1,\ldots,n
\]

В соответствии с определением интерполяционного полинома получаем:
\[
    \displaystyle \sum_{k = 0}^{n} c_k(x_i)y_k = y_i, \quad c_k(x_i) =  
    \begin{cases}
        0 & i \neq k \\
        1 & i = k
    \end{cases}, \quad i = 0,1,\ldots,n
\]

\subsection{Кубический сплайн}

Кубическим сплайном для функции $y(x)$ называют функцию $S(x)$, удовлетворяющую следующим условиям:

\begin{enumerate}
    \item на каждом отрезке $ [x_{i - 1}, x_{i}] $ функция $S(x)$ –- многочлен третьей степени;
    \item функция $S(x)$, ее первая и вторая производные непрерывны на отрезке $[x_0, x_n]$;
    \item значения функции $S(x)$ и исходной функции $y(x)$ совпадают в узлах интерполяции.
\end{enumerate}

На каждом из отрезков $ [x_{i - 1}, x_{i}] $ функция $S(x) = s_i$ ищется следующим образом 
\[
    s_i = a_i + b_i(x - x_i) + c_i(x - x_i)^2 + d_i(x - x_i)^3 = a_i + b_i h_i + c_i h_i^2 + d_i h_i^3, 
\]

\noindent где $a_i, b_i, c_i, d_i $ -- коэффициенты, подлежазие определению.
\[
    \begin{split}
        & a_i = y_{i - 1}\\
        & a_i + b_i h_i + c_i h_i^2 + d_i h_i^3 = y_i
    \end{split}
\]

Из условия непрерывности первой и второй производной получаем
\[
    \begin{split}
        S^'(x_i - 0) & = S^'(x_i + 0)  \\
        S^{''}(x_i - 0) & = S^{''}(x_i + 0), \quad i = 1, 2, \ldots, n - 1,
    \end{split} 
\]

\noindent тогда 
\[
    \begin{split}
        b_i + 2 c_i h_i + 3 d_i h_i^2 & = b_{i + 1} \\
        2 c_i + 6 d_i h_i & = 2 c_{i + 1}
    \end{split} 
\]

\pagebreak

Положим $ S^{''}(x_0) = S^{''}(x_n) = 0 $, тогда
\[
    \begin{split}
        &2 c_1 = 0 \\ 
        &2 c_n + 6 d_n h_n = 2 c_{n + 1} = 0
    \end{split}   
\]

Введением вспомогательного параметра $ g_i = \dfrac{y_i - y_{i-1}}{h_i} $ получаем систему
\[
  \begin{cases}
    c_1 = 0 \\
    h_{i - 1} c_{i - 1} + 2 (h_{i - 1} + h_i) c_i + h_i c{i + 1} = 3(g_i - g{i - 1}) \\
    c_{n + 1} = 0,
  \end{cases}  
\]

\noindent которая является трехдиагональной и обладает диагональным преобладанием, поэтому для нахождения коэффициентов можно использовать метод прогонки.

Остальные коэффициенты находим по формулам
\[
    \begin{split}
        b_i &= g_i  - \dfrac{(c_{i + 1} + 2 c_i) h_i}{3} \\ 
        d_i &= \dfrac{c_{i + 1} - c_i}{3 h_i}, \quad i = 1, 2, \ldots, n
    \end{split} 
\]


\section{Исходные данные}

\newpage

\section{Результаты расчетов}

\newpage

\section{Контрольные вопросы}
\begin{enumerate}

\item Определите количество арифметических операций, требуемое для интерполирования функции в некоторой точке многочленом Лагранжа (включая построение самого многочлена) на сетке с числом узлов, равным $n$.

Для того, чтобы посчитать коэффициент $c_k(x)$ и умножить его на $y_k$, нужно $n$ операций. Для подсчета суммы всех произведений $k = \overline{1,..., n}$ нужно $n \cdot n = n^2$ операций.

\item Определите количество арифметических операций, требуемое для интерполирования функции в некоторой точке кубическим сплайном (включая затраты на вычисление коэффициентов сплайна) на сетке с числом узлов, равным $n$.

Для подсчета $g_i$ нужно $n$ операций. Для прогонки потребуется $5n$ операций. Далее для подсчета коэффициентов $b_i$ и $d_i$ нужно $3n + 2n = 5n$ операций. Итог: $n + 5n + 5n = 11n$.

\pagebreak

\item Функция $f(x) = e^x$ интерполируется многочленом Лагранжа на отрезке $[0, 2]$ на равномерной сетке с шагом $h = 0{.}2$. Оцените ошибку экстраполяции в точке $x = 2{.}2$, построив многочлен Лагранжа и подставив в него это значение, а также по формуле для погрешности экстраполяции.

\begin{figure}[H]
    \center{\includegraphics[scale=0.7]{pic/misclosure.png}}
    \caption{Полином Лагранжа для функции $Exp[x]$. Нахождение значения этого полинома в точке $ x = 2.2 $. Сравнение с оценочной формулой}
    \label{fig:misclosure}
\end{figure}

\pagebreak

\item Выпишите уравнения для параметров кубического сплайна, если в узлах $x_0$ и $x_n$ помимо значений функции $y_0$ и $y_n$ заданы первые производные $y'(x_0)$ и $y'(x_n)$.

\item Каковы достоинства и недостатки сплайн-интерполяции и интерполяции многочленом Лагранжа?

\item Какие свойства полиномов Чебышева и чебышевских сеток Вам известны?


\end{enumerate}
\newpage

\end{document} 
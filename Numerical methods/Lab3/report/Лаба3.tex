\documentclass[12pt, a4paper]{article}

\usepackage[utf8]{inputenc}
\usepackage[T2A]{fontenc}
\usepackage[russian]{babel}
\usepackage[]{float}

\usepackage[oglav, boldsect, eqwhole, figwhole, %
   remarks, hyperref, hyperprint]{fn2kursstyle}

\frenchspacing
\righthyphenmin=2

%Командна для римских прописных 
\newcommand{\RomanNumeralCaps}[1]
    {\MakeUppercase{\romannumeral #1}}

\title{Решение задач интерполирования}
\group{ФН2-52Б}
\author{A.\,И.~Токарев}
\secauthor{Ю.\,А.~Сафронов}
\supervisor{}
\date{2021}

\def\hmath$#1${\texorpdfstring{{\rmfamily\textit{#1}}}{#1}}

\begin{document}
\maketitle
\tableofcontents 
\newpage

\section{Краткое описание алгоритмов}



\section{Исходные данные}

\newpage

\section{Результаты расчетов}

\newpage

\section{Контрольные вопросы}
\begin{enumerate}

\item Определите количество арифметических операций, требуемое для интерполирования функции в некоторой точке многочленом Лагранжа (включая построение самого многочлена) на сетке с числом узлов, равным $n$.

Для того, чтобы посчитать коэффициент $c_k(x)$ и умножить его на $y_k$, нужно $n$ операций. Для подсчета суммы всех произведений $k = \overline{1,..., n}$ нужно $n \cdot n = n^2$ операций.

\item Определите количество арифметических операций, требуемое для интерполирования функции в некоторой точке кубическим сплайном (включая затраты на вычисление коэффициентов сплайна) на сетке с числом узлов, равным $n$.

Для подсчета $g_i$ нужно $n$ операций. Для прогонки потребуется $5n$ операций. Далее для подсчета коэффициентов $b_i$ и $d_i$ нужно $3n + 2n = 5n$ операций. Итог: $n + 5n + 5n = 11n$.

\item Функция $f(x) = e^x$ интерполируется многочленом Лагранжа на отрезке $[0, 2]$ на равномерной сетке с шагом $h = 0{.}2$. Оцените ошибку экстраполяции в точке $x = 2{.}2$, построив многочлен Лагранжа и подставив в него это значение, а также по формуле для погрешности экстраполяции.

\item Выпишите уравнения для параметров кубического сплайна, если в узлах $x_0$ и $x_n$ помимо значений функции $y_0$ и $y_n$ заданы первые производные $y'(x_0)$ и $y'(x_n)$.

\item Каковы достоинства и недостатки сплайн-интерполяции и интерполяции многочленом Лагранжа?

\item Какие свойства полиномов Чебышева и чебышевских сеток Вам известны?


\end{enumerate}
\newpage

\end{document} 
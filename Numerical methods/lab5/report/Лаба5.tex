\documentclass[12pt, a4paper]{article}

\usepackage[utf8]{inputenc}
\usepackage[T2A]{fontenc}
\usepackage[russian]{babel}
\usepackage[]{float}

\usepackage[oglav, boldsect, eqwhole, figwhole, %
   remarks, hyperref, hyperprint]{fn2kursstyle}

\frenchspacing
\righthyphenmin=2

%Командна для римских прописных 
\newcommand{\RomanNumeralCaps}[1]
    {\MakeUppercase{\romannumeral #1}}

\title{Методы решения нелинейных уравнений}
\group{ФН2-52Б}
\author{A.\,И.~Токарев}
\secauthor{Ю.\,А.~Сафронов}
\supervisor{}
\date{2021}

\def\hmath$#1${\texorpdfstring{{\rmfamily\textit{#1}}}{#1}}

\begin{document}
\maketitle
\tableofcontents 
\newpage

\section{Краткое описание алгоритмов}
\subsection{Локализация корней}

Дано нелинейное уравнение $f(x) = 0, \quad x \in [a,\, b]$, требуется найти все отрезки принадлежащие $[a,\, b]$, на которых уравнение имеет единственный корень, т. е. произвести локализацию корней. Для этого воспользуемся первой теоремой Больцано---Коши из классического анализа:

если непрерывная на отрезке $[a,\, b]$ функция $f(x)$, такая , что $f(a)f(b) < 0$, то $\exists c \in [a,\, b] \colon f(c) = 0$.

Соответственно, чтобы локализовать корни, нужно составить достаточно подробное дробление отрезка  $[a,\, b]$ и проверить на каждом из них условие теоремы. Составим дробление отрезка: $a_1 = a, a_2 = a + h, ... a_{n-1} = b - h, a_n = b$, где $h$ --- шаг дробления, а $n$ --- количество точек, включая концы отрезка.

\subsection{Метод бисекций}
Пусть теперь $[a_i,\, a_{i+1}] \subset [a,\, b]$ --- отрезок локализации $f(x)$, т.е. $f(a_i)f(a_{i+1})<0$. Найдем корень уравнения $f(x)=0$ на $i$-ом отрезке локализации с наперед заданной точностью $\varepsilon$.

\begin{enumerate}
\item Обозначим $\alpha_0 = a_i$, $\beta_0 = a_{i+1}$, тогда $x_0 = \dfrac{\beta_0 - \alpha_0}{2}$;
\item Если $\left|\dfrac{\beta_0  - \alpha_0}{2} \right| < 2\varepsilon$, то корень найден,  $x^* = x_0$, иначе идем к пункту 3;
\item Если $f(x_0)f(\alpha_0) < 0$, то  $\alpha_1 = \alpha_0$, $\beta_1 = x_0$. Если $f(x_0)f(\beta_0) < 0$, то $\alpha_1 = x_0$, $\beta_1 = \beta_0$. Тогда $x_1 = \dfrac{\beta_1 - \alpha_1}{2}$;
\item Если $\left|\dfrac{\beta_0  - \alpha_0}{4} \right| < 2\varepsilon$, то корень найден $x^* = x_1$, иначе повторяем процедуру из пункта 3 для последующих значений $x_1, x_2, ... x_k$ до тех пор, пока не выполнится условие $\left|\dfrac{\beta_k  - \alpha_k}{2^{k-1}} \right| < \varepsilon$. Тогда $x^* = x_k$.
\end{enumerate}

Данный метод повторяем для всех отрезков локализации. 

\subsection{Метод Ньютона}

Разложим функцию $f(x)$ в ряд Тейлора в окрестности известного приближения корня $x_k$, пренебрегая величинами больше второго порядка малости и принимая истинное значение корня за $x_{k+1}$, тогда уравнение примет вид
\[
f(x_k) + f^'(x_k)(x_{k+1} - x_k) = 0,
\]
отсюда получим итерационную формулу метода Ньютона:
\[
x_{k+1} = x_k - \dfrac{f(x_k)}{f^'(x_k)}, \quad k=0,1,2...
\]


\section{Исходные данные}

Вариант 20: интервал --- $[-1,\, 0]$
 
$f(x) = \sin\left(\dfrac{x^3\sqrt{13} - 9x - 5 - \sqrt{17}}{10}\right) + \tan\left({\dfrac{x^2 + x + 2^{\frac{1}{3}}}{3x - 5}}\right) + 0{.}6$.

Вариант 23: интервал --- $[-1,\, 0]$
         
$f(x) = \sin\left(\dfrac{-2x^2 - x\sqrt{10} + 1}{4}\right)+ \left(\dfrac{x^2+x(\sqrt{2}+ \sqrt{7}) + 1 - \sqrt{5}}{x\sqrt{7} - \sqrt{5}}\right)^{\ln{2}}  - 0{.}1$.

\section{Результаты расчетов}


\section{Контрольные вопросы}
\begin{enumerate}
\item Можно ли использовать методы бисекции и Ньютона для нахождения кратных корней уравнения $f(x) = 0$ (т. е. тех, в которых одна или несколько первых производных функций $f(x)$ равны нулю)? Обоснуйте ответ.

Метод бисекции можно использовать для абсолютно любой функции в любых ситуациях, так как он не использует никакую информацию о функции, кроме значения в точках. На отрезке локализации он найдет корень, но не всегда быстро. Метод Ньютона не сойдется при условии, что первая производная в корне равна нулю в силу построения итерационного метода. Если производная в точке корня равна нулю, то скорость сходимости не будет квадратичной, а сам метод может преждевременно прекратить поиск, и дать неверное для заданной точности приближение. Поскольку для кратных корней как минимум первая производная обращается в ноль, то методом Ньютона такие корни не найти.

\item При каких условиях можно применять метод Ньютона для поиска корней уравнения $f(x) = 0$? При каких ограничениях на функцию $f(x)$ метод Ньютона обладает квадратичной скоростью сходимости? В каких случаях можно применять метод Ньютона для решения систем нелинейных уравнений?

Если в некоторой окрестности корня $x^*$ выполнены условия $|f^'(x)| > m > 0$, $|f^{''}(x)|< M$, $\dfrac{|f(x)F^{''}(x)|}{(f^'(x)))^2} < 1$, где $m$, $M$ --- константы, то при попадании очередного приближения $x_s$ в эту окрестность итерационный процесс по методу Ньютона будет сходиться с квадратичной скоростью$\colon |x_{k+1} - x^*| < C |x_k - x^*|^2,\quad k = s, s + 1, s + 2, ...$



\item Каким образом можно найти начальное приближение?

\item Можно ли использовать метод Ньютона для решения СЛАУ?

\item Предложите альтернативный критерий окончания итераций в методе бисекции, в котором учитывалась бы возмоность попадания очередного приближения в очень малую окрестность корня уравнения.

\item Предложите различные варианты модификаций метода Ньютона. Укажите их достоинства и недостатки.

\item Предложите алгоритм для исключения зацикливания метода Ньютона и выхода за пределы области поиска решения?
\end{enumerate}

\end{document} 